\section*{Annexe B -- Protocole de test reproductible}

Cette annexe décrit la procédure minimale permettant de reproduire les tests présentés dans ce rapport.

\subsection*{B.1 Préparation}

\begin{enumerate}
  \item Se placer dans le dossier du serveur de test.
  \item Installer les dépendances Node.js.
  \item Démarrer le serveur local (port 3000).
\end{enumerate}

\subsection*{B.2 Configuration de l'extension}

\begin{enumerate}
  \item Ouvrir la page d'administration des extensions du navigateur Chromium.
  \item Activer le mode développeur.
  \item Charger le projet en mode \textit{Load unpacked}.
  \item Dans le popup, renseigner :
        \begin{itemize}
          \item URL serveur : \texttt{http://localhost:3000}
          \item Clé API : \texttt{trafficapikey}
        \end{itemize}
\end{enumerate}

\subsection*{B.3 Exécution et collecte}

\begin{enumerate}
  \item Cliquer sur \textit{Connect} pour démarrer le polling.
  \item Générer du trafic en naviguant vers des routes locales.
  \item Observer les métriques agrégées et le graphe des latences.
  \item Utiliser \textit{Refresh} pour forcer une mise à jour immédiate.
\end{enumerate}

\subsection*{B.4 Critères de validation}

Le test est considéré valide si :
\begin{itemize}
  \item des entrées apparaissent dans le flux trafic ;
  \item les compteurs évoluent avec la charge générée ;
  \item les requêtes lentes sont correctement identifiées ;
  \item le test de santé (\texttt{/api/health}) confirme la connectivité.
\end{itemize}