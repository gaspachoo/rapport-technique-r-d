\section{Résultats}
\label{sec:resultats}

\subsection{Résultats fonctionnels}

Le prototype atteint les objectifs fonctionnels fixés pour une première itération :

\begin{itemize}
    \item connexion à une API de trafic avec authentification par clé ;
    \item affichage en continu des requêtes récentes (méthode, endpoint, statut, durée, horodatage) ;
    \item calcul et affichage d'indicateurs agrégés de performance ;
    \item visualisation de l'évolution des temps de réponse sous forme de graphe ;
    \item test de connectivité via endpoint de santé.
\end{itemize}

La fréquence de polling retenue permet une perception quasi temps réel de l'activité serveur, avec une interface suffisamment compacte pour un usage en contexte de diagnostic rapide.

\subsection{Capacité de diagnostic}

Les expérimentations sur serveur local mettent en évidence trois apports principaux :

\begin{itemize}
    \item \textbf{Détection des lenteurs :} les pics de durée sont rapidement visibles dans le graphe et la liste des requêtes les plus lentes ;
    \item \textbf{Identification des zones chaudes :} les endpoints les plus sollicités sont isolés, facilitant la priorisation des optimisations ;
    \item \textbf{Lecture de la qualité de service :} la combinaison taux de succès/taux d'erreur fournit un indicateur simple de stabilité.
\end{itemize}

Ces observations confirment l'intérêt d'une agrégation continue, complémentaire aux audits ponctuels réalisés par des outils de benchmark \cite{afonso2020correlating,bajaj2019hybrid}.

\subsection{Bilan par objectif}

\begin{center}
    \begin{tabular}{|p{6.5cm}|p{3.8cm}|}
        \hline
        \textbf{Objectif}                               & \textbf{Niveau d'atteinte} \\
        \hline
        Identifier des indicateurs clés de performance  & Atteint                    \\
        \hline
        Concevoir un protocole de mesure simple         & Atteint                    \\
        \hline
        Développer un prototype d'extension fonctionnel & Atteint                    \\
        \hline
        Tester sur différents types de sites            & Partiellement atteint      \\
        \hline
        Proposer une visualisation claire des résultats & Atteint                    \\
        \hline
    \end{tabular}
\end{center}

Le point partiellement atteint concerne la diversité des environnements testés. À ce stade, l'évaluation repose principalement sur un serveur local de simulation et non sur une campagne exhaustive multi-sites.