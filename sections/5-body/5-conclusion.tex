\section{Conclusion}
\label{sec:conclusion}

Ce projet a permis de concevoir et d'implémenter un prototype d'extension navigateur capable de mesurer et d'analyser des performances web à partir de données de trafic HTTP. Les objectifs principaux de la note de cadrage sont globalement atteints : définition d'indicateurs pertinents, mise en œuvre d'un protocole de mesure simple, réalisation d'un outil fonctionnel et proposition d'une restitution visuelle exploitable.

Les résultats confirment la pertinence d'une approche légère, embarquée dans le navigateur, pour un diagnostic rapide des ralentissements. Le prototype facilite l'identification des points de tension (endpoints lents, hausse d'erreurs, variations de latence) et constitue une base crédible pour une démarche d'amélioration continue.

Les perspectives de travail sont les suivantes :
\begin{itemize}
    \item élargir la campagne de tests à des sites variés et à des scénarios de charge plus réalistes ;
    \item intégrer davantage de métriques orientées utilisateur final (Core Web Vitals complets) ;
    \item enrichir la comparaison avec des outils de référence afin d'évaluer la robustesse des mesures ;
    \item améliorer les mécanismes d'explication pour renforcer la valeur pédagogique auprès d'utilisateurs non experts.
\end{itemize}

Dans l'ensemble, le projet démontre qu'une extension navigateur peut constituer un support pertinent pour relier performance, observabilité et sensibilisation aux enjeux techniques et environnementaux du web.