\section{Matériel et méthodes}
\label{sec:materiel-methodes}

\subsection{Démarche générale}

La démarche suivie reprend les étapes de la note de cadrage : revue de littérature, sélection d'indicateurs, conception de l'architecture, développement d'un prototype, puis phase de test et d'analyse. Le choix a été fait de privilégier une approche incrémentale avec un périmètre fonctionnel réduit mais opérationnel.

\subsection{Architecture du prototype}

Le prototype est implémenté sous forme d'extension Chromium en JavaScript. Son fonctionnement repose sur trois briques principales :

\begin{itemize}
    \item une interface utilisateur (popup) pour la configuration, l'affichage du trafic et la visualisation des indicateurs ;
    \item une logique d'analyse agrégée pour extraire les tendances (endpoints dominants, taux de succès, lenteurs) ;
    \item une API serveur exposant les flux de trafic et métriques, interrogée en \textit{polling}.
\end{itemize}

L'extension interroge \texttt{/api/traffic} toutes les 100 ms en mode connecté, puis met à jour les cartes de synthèse et le graphe des temps de réponse. Un endpoint de santé (\texttt{/api/health}) est utilisé pour valider la connectivité. Les données d'authentification (clé API) sont stockées localement via \texttt{chrome.storage.local}. Le détail du contrat d'API est fourni en \hyperref[annexe:api]{Annexe A}.

\subsection{Indicateurs retenus}

La sélection des indicateurs vise un compromis entre simplicité de compréhension et utilité technique. Les mesures présentées dans l'interface sont :

\begin{itemize}
    \item temps de réponse moyen et maximal ;
    \item taux de succès et taux d'erreur ;
    \item volume total de requêtes et rythme de requêtes par minute ;
    \item répartition des méthodes HTTP et des codes de statut ;
    \item classement des endpoints les plus sollicités et des requêtes les plus lentes.
\end{itemize}

Ces indicateurs s'inscrivent dans la continuité des recommandations issues de la littérature sur la performance perçue, les tests de charge et la planification de capacité \cite{menasce2002capacity,butkiewicz2011understanding,ionescu2015comparative}.

\subsection{Protocole expérimental}

Un serveur local de test (Node.js/Express) a été utilisé pour simuler le trafic entrant et exposer les endpoints minimaux : \texttt{/api/traffic}, \texttt{/api/metrics} et \texttt{/api/health}. Le protocole comprend :

\begin{enumerate}
    \item démarrage du serveur de test et connexion de l'extension ;
    \item génération manuelle de trafic via des requêtes HTTP répétées ;
    \item observation en temps réel des évolutions de métriques ;
    \item vérification de la cohérence des agrégations affichées.
\end{enumerate}

En complément, une interface de signaux de sécurité de type slowloris a été intégrée. Les endpoints associés (\texttt{/api/slowloris}, \texttt{/api/block-ip}, \texttt{/api/unblock-ip}, \texttt{/api/blocked-ips}) restent optionnels et ne sont pas fournis par le serveur de test standard. La procédure de test reproductible est détaillée en \hyperref[annexe:protocole]{Annexe B}.